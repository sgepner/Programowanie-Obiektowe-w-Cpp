\documentclass[10pt]{beamer}

\usepackage[english,polish]{babel}
\usepackage[utf8]{inputenc}
\usepackage{polski}

\usepackage{define}

\eventtitle{Programowanie obiektowe w języku C++}
\title{Programowanie obiektowe w języku C++}
\author[shortname]{Stanis{\l}aw Gepner}
\institute[shortinst]{sgepner@meil.pw.edu.pl}
\date{}

\setbeamertemplate{blocks}[rounded][shadow=true]
\setbeamertemplate{navigation symbols}[]

\begin{document}

\frame{
    \titlepage
}

\section{Wyjątki}

\begin{frame}
  \frametitle{Wyjątki}
  
  \begin{itemize}
		\item Metoda przeniesienia kontroli do w przypadku wystąpienia wyjątkowych sytuacji
		\item Wykorzystujemy blok \textit{try\{\}catch()\{\}}
		\item Wyjątek podniesiony w sekcji \textit{try} przenosi kontrolę do odpowiedniej sekcji \textit{catch()}
  \end{itemize}
  \vspace{0.2cm}
  \begin{tikzpicture}[scale=2, node distance = 2cm, auto]
    % Place nodes
    \node [circle, draw, fill=black, minimum size=0.2cm] (A) at (0,0) {};
    \node [circle, draw, fill=black, minimum size=0.2cm] (END) at (5,0) {};
    \node [] (END2) at (5.5,0) {};
    \node [rectangle, draw, fill=black, minimum size=0.2cm, text=white,aspect=2] (B) at (0.5,0) {try\{};
    \path [draw, -latex, thick] (A) -- (B);
    \node [rectangle, draw, fill=black, minimum size=0.2cm, text=white,aspect=2] (C) at (1.3,0) {\parbox{1.5cm}{my\\ protected code}};
    \path [draw, -latex, thick] (B) -- (C);
    \node [rectangle, draw, fill=black, minimum size=0.2cm, text=white,aspect=2] (D) at (2.5,0) {\parbox{2cm}
    {\centering tu się mogą dziać złe rzeczy!\\ \textbf{throw}}};
    \path [draw, -latex, thick] (C) -- (D);
    \node [rectangle, draw, fill=black, minimum size=0.2cm, text=white,aspect=2] (E) at (3.7,0) {\parbox{1.5cm}{my\\ protected code}};
    \path [draw, -latex, thick] (D) -- (E);
    \node [rectangle, draw, fill=black, minimum size=0.2cm, text=white,aspect=2] (F) at (4.4,0) {\}};
    \path [draw, -latex, thick] (E) -- (F);
    \path [draw, -latex, thick] (F) -- (END);
    \path [draw, -latex, thick] (F) -- (END2);
    
    \node [rectangle, draw, fill=black, minimum size=0.2cm, text=white,aspect=2] (A) at (2,-1) {catch(arg)\{};
    \path [draw, -latex, thick] (D) -- (A);
    \node [rectangle, draw, fill=black, minimum size=0.2cm, text=white,aspect=2] (B) at (3.5,-1) {Obsługuję wyjątek};    
    \path [draw, -latex, thick] (A) -- (B);
    \node [rectangle, draw, fill=black, minimum size=0.2cm, text=white,aspect=2] (C) at (4.6,-1) {\}};
    \path [draw, -latex, thick] (B) -- (C);
    \path [draw, -latex, thick] (C) -| (END);

\end{tikzpicture}
\end{frame}

\begin{frame}[fragile]
  \frametitle{Wyjątki}
  
\begin{itemize}
	\item można rzucać prawie wszystkim
	\item a łapać, określony typ, ale nie tylko
\end{itemize}  
  
\begin{lstlisting}
#include <iostream>
using namespace std;

int main()
{
  try
  {
    throw 2;
  }
  catch (int e)
  {
    cout << "An exception occurred. Exception Nr. " << e << '\n';
  }
  return 0;
}
\end{lstlisting}

\end{frame}

\begin{frame}[fragile]
  \frametitle{Wyjątki}
  
\begin{itemize}
	\item \textit{catch(...)\{} złapie 'co kolwiek'
\end{itemize}  
  
\begin{lstlisting}
#include <iostream>
#include <string>
using namespace std;
int main(){
  try  {
  	int a;
  	cin >> a;
  	switch(a)
  	{
  		case 1:
  			throw 5;
  		case 2:
  			throw 'a';
  		case 3:
  			throw string("String");
  		case 4:
  			throw 5.0;
  	}
  }
  catch (int e){ cout << "An int: " << e <<endl;}
  catch (char e){ cout << "Char: " << e << endl;}
  catch (string e){ cout << "String " << e << endl;}
  catch (...){ cout << " Default exception!!" << endl;}
  return 0;}
\end{lstlisting}
\end{frame}

\begin{frame}[fragile]
  \frametitle{Wyjątki}
  
\begin{itemize}
	\item Można zagnieżdżać
\end{itemize}  
  
\begin{lstlisting}
try{
	try{
		//Code that does some serious stuff
		throw 4;
	}
	catch (int e){
		cout << "App thrown an exeption, terhrowing!" <<endl;
		ostringstream os;
		os << "Value is: " << e;
		throw os.str();
	}
}
catch (string e)
{
  cout << "An exception occurred. String " << e << endl;
}
catch (...)//domyslnie
{
	cout << " Default exception!!" << endl;
}
\end{lstlisting}
\end{frame}

\subsection{exception}

\begin{frame}[fragile]
  \frametitle{exception}
  
\begin{itemize}
	\item $include <exception>$
\end{itemize}  
  
\begin{lstlisting}
class exception {
public:
  exception () throw();
  exception (const exception&) throw();
  exception& operator= (const exception&) throw();
  virtual ~exception() throw();
  virtual const char* what() const throw();
}
\end{lstlisting}
\end{frame}

\begin{frame}[fragile]
  \frametitle{exception}
\begin{itemize}
	\item Możemy tworzyć własne!
\end{itemize}  
  
\begin{lstlisting}
class myexception: public exception
{
public:
	myexception(const string& s): message(s) {}
	virtual ~myexception() throw(){}
private:
	string message;
  virtual const char* what() const throw()
  {
    return message.c_str();
  }
};
\end{lstlisting}

\end{frame}

\section{Wątki i procesy}

\begin{frame}
	\frametitle{Proces vs Wątek}
	\framesubtitle{Gdy jest nas wielu}
	\begin{columns}
	\begin{column}{0.5\textwidth}
	\centering \textbf{Process}
	\begin{itemize}
		\item Niezależny
		\item Własna, niedzielona przestrzeń adresowa
		\item Komunikacja tylko przez system (shared memory, semaphores)
	\end{itemize}
	
	\centering \textbf{Wątki}
	\begin{itemize}
		\item Istnieje w procesie
		\item Wspólna przestrzeń adresowa
	\end{itemize}
	\end{column}
	\begin{column}{0.5\textwidth}
	\centering
  \begin{tikzpicture}[scale=2, auto]
    % Place nodes
    \node [circle, draw, fill=blue!50, minimum size=3cm] (A) at (0,0) {};
    \node [font=\small, anchor=south] at ($(A)+(0,0.4)$) {Process 1};
		\path [draw, decorate,decoration=snake, thick] (-0.4,-0.5) -- node [] {\tiny T1} (-0.4,0.4);
		\path [draw, decorate,decoration=snake, thick] (-0.1,0.1) -- node [] {\tiny T2} (-0.1,-0.4);
		\path [draw, decorate,decoration=snake, thick] (0.35,-0.2) -- node [] {\tiny T3} (0.35,0.2);
    \node [circle, draw, fill=blue!50, minimum size=3cm] (A) at (0,-1.6) {};
    \node [font=\small, anchor=south] at ($(A)+(0,0.4)$) {Process 2};
		\path [draw, decorate,decoration=snake, thick] (-0.5,-1.8) -- node [] {\tiny T1} (-0.5,-1.2);
		\path [draw, decorate,decoration=snake, thick] (-0.1,-2.1) -- node [] {\tiny T2} (-0.1,-1.4);
\end{tikzpicture}
	\end{column}
	\end{columns}
\end{frame}

\begin{frame}[fragile]
	\frametitle{Wątek}
	\framesubtitle{Nowy wątek}
	\begin{columns}
	\begin{column}{0.5\textwidth}
\begin{lstlisting}
#include <thread>

void foo(){
  for(int i=0; i<10; ++i){
  	cout << "foo says: foo sleeps " << i << endl;
  	sleep ( 1 );
  }
}
void bar(int x){
  for(int i=0; i<10; ++i){
  	cout << "bar says: x=" << x << " bar sleeps " << i << endl;
  	sleep ( 1 );
  }
}
...
std::thread first (foo);
std::thread second (bar,0);

\end{lstlisting}
	\end{column}
	\begin{column}{0.5\textwidth}
	\centering
\begin{tikzpicture}[scale=2, node distance = 2cm, auto]
    % Place nodes
    \node [circle, draw, fill=black, minimum size=0.4cm,inner sep=0pt, outer sep=0pt] (A) at (0,0) {};
    \node [circle, draw, fill=black, minimum size=0.2cm,inner sep=0pt, outer sep=0pt] (AA) at (0,-0.3) {};
    \node [] (B) at (0,-1.5) {};
    \node [] (C1) at (-1,-1.5) {};
    \node [] (C2) at (1,-1.5) {};
    \path [draw, -latex, very thick] (A) -- node [] (C) {} (B);
    \node [rotate=-90] at ($(C)+(0.05,-0.05)$) {\small main thread};
    \path [draw, -latex] (AA) -| node [near start] (C) {}(C1);
    \node [] at ($(C)+(0.0,0.15)$) {\small \textit{foo()}};
    \path [draw, -latex] (AA) -| node [near start] (C) {}(C2);
    \node [] at ($(C)+(0.0,0.03)$) {\small \textit{bar(0)}};
%    \node [] (C) at (1,-0.7) {};
%    \path [draw, -latex, thick] (B) -| node [near start] {yes} (C);
%    \path [draw, -latex, thick] (C) |- (AA);
%    \node [inner sep=0pt, outer sep=0pt] (BB) at (0,-2.5) {};
%    \path [draw, -latex, thick] (B) -> node [near start] {no} (BB);
\end{tikzpicture}
	\end{column}
	\end{columns}
\end{frame}

\begin{frame}[fragile]
	\frametitle{Wątki}
	\framesubtitle{\textit{join detach}}
	\begin{columns}
	\begin{column}{0.5\textwidth}
\begin{lstlisting}
#include <thread>

...
std::thread first (foo);
std::thread second (bar,0);
// detach first from main
first.detach()
...
//main thread does sth
...
// wait for second to terminate
second.join()



\end{lstlisting}
	\end{column}
	\begin{column}{0.5\textwidth}
	\centering
\begin{tikzpicture}[scale=2, node distance = 2cm, auto]
    % Place nodes
    \node [circle, draw, fill=black, minimum size=0.4cm,inner sep=0pt, outer sep=0pt] (A) at (0,0) {};
    \node [circle, draw, fill=black, minimum size=0.2cm,inner sep=0pt, outer sep=0pt] (AA) at (0,-0.3) {};
    \node [] (B) at (0,-2.5) {};
\tikzset{cross/.style={cross out, draw=black, minimum size=1*(#1-\pgflinewidth),
inner sep=0pt, outer sep=0pt, very thick}, cross/.default={10pt}}
    \node [draw,circle,cross,red] (C1) at (-1,-1.5) {};
    \node [] at ($(C1)+(0.0,-0.15)$) {\small first terminates};    
    \node [inner sep=0pt, outer sep=0pt] (C2) at (1,-1.8) {};
    \path [draw, -latex, very thick] (A) -- node [] (C) {} (B);
    \node [rotate=-90] at ($(C)+(0.05,0.4)$) {\small main thread};
    \path [draw, -latex] (AA) -| node [near start] (C) {}(C1);
    \node [] at ($(C)+(0.0,0.15)$) {\small \textit{foo()}};
    \path [draw, ] (AA) -| node [near start] (C) {}(C2);
    \node [] at ($(C)+(0.0,0.03)$) {\small \textit{bar(0)}};
    \node [circle, draw, fill=black, minimum size=0.2cm,inner sep=0pt, outer sep=0pt] (AA) at (0,-1.8) {};
    \node [] at ($(AA)+(0.4,-0.2)$) {\parbox{1.5cm}{\centering \small main waits for second}};
    \path [draw, -latex] (C2) -- (AA);

\end{tikzpicture}
	\end{column}
	\end{columns}
\end{frame}

\begin{frame}[fragile]
	\frametitle{Wątki}
	\framesubtitle{Dostęp do zasobów}

\begin{itemize}
	\item Wątki współdzielą pamięć (i inne)
	\item Należy unikać 'wyścigu'
	\item \textit{mutex}
	\item \textit{lock} i \textit{unlock}
\end{itemize}

\end{frame}

\section{Process}

\begin{frame}[fragile]
	\frametitle{Procesy}
	\framesubtitle{fork}
  \begin{itemize}
    \item fork
    \item morph
  \end{itemize}


\end{frame}

\end{document}
