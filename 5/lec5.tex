%\documentclass[draft,10pt]{beamer}
\documentclass[10pt]{beamer}

\usepackage[english,polish]{babel}
\usepackage[utf8]{inputenc}
\usepackage{polski}

\usepackage{define}
\usepackage{tikz-uml}

\eventtitle{Programowanie obiektowe w języku C++}
\title{Programowanie obiektowe w języku C++}
\author[shortname]{Stanis{\l}aw Gepner}
\institute[shortinst]{sgepner@meil.pw.edu.pl}
\date{}

\setbeamertemplate{blocks}[rounded][shadow=true]
\setbeamertemplate{navigation symbols}[]

\begin{document}

\frame{
    \titlepage
}

\section{Polimorfizm - dokończenie}

\begin{frame}
  \frametitle{Polimorfizm - dokończenie}
  \framesubtitle{Lepsza farma}
  
  \begin{columns}
    \begin{column}{0.5\textwidth}
      \begin{tikzpicture}
        \umlsimpleclass{bydle}
        \umlsimpleclass[x=-2, y=-2]{krowa}
        \umlsimpleclass[x=-0.3, y=-2]{kon}
        \umlsimpleclass[x=1, y=-3]{osiol}
        \umlsimpleclass[x=1.8, y=-2]{kura}
        \umlsimpleclass[x=-1, y=-4]{mul}
        \umlinherit{krowa}{bydle}
        \umlinherit{kon}{bydle}
        \umlinherit{osiol}{bydle}
        \umlinherit{kura}{bydle}
        \umlinherit{mul}{kon}
        \umlinherit{mul}{osiol}
      \end{tikzpicture}
    \end{column}
    \begin{column}{0.6\textwidth}
      \begin{itemize}
        \item Polimorfizm - czyli wielopostaciowość
        \item Zdolność kodu do różnego zachowania w trakcie wykonania
        \item Wskaźnik polimorficzny \textit{bydle *p = ...}
        \item ... tzn. taki, który możemy róznie interpretować
        \item To funkcje składowe klasy mogą być polimorficzne,
        \item a nie obiekt klasy
        \item Określenie, która funkcja jest wykonana dla danego obiektu jest przesunięte do czasu wykonania
      \end{itemize}
    \end{column}
  \end{columns}
\end{frame}

\section{Szablony - template}

\begin{frame}[fragile]
  \frametitle{Szablony}
  \framesubtitle{Jak pisać mniej a dostać więcej}
  \begin{columns}
    \begin{column}{0.6\textwidth}
    \begin{itemize}
      \item Szablon, \textit{templejt*}, generyk ...
      \item Umożliwia tworzenie kodu niezależnego od typów, większa elastyczność
      \item Pozwala na nieduplikowanie kodu
      \item Wykonanie w czasie kompilacji, jak makra preprocesora C
      \item Szablony funkcji, klas, metod
      \item Szablon może być oparty o typ (klasę) lub wartość całkowitą
      \item Metaprogramowanie: w oparciu o mechanizm szablonu. Tak język szablonów jest kompletny w sensie Turinga.
    \end{itemize}
    \end{column}
    \begin{column}{0.47\textwidth}
\begin{lstlisting}
template <class T> 
typ_funkcji fun(argumenty)
{
  Cialo funkcji
}

template <class T> 
class List
{
    ...
};
\end{lstlisting}
{\tiny *takiego słowa nie ma ...}
    \end{column}
  \end{columns}
\end{frame}

\subsection{Funkcje}

\begin{frame}[fragile]
  \frametitle{Bez szablonów funkcji}
  \framesubtitle{Po co? przykład z przeładowaniem}

Tworzymy funkcję \textit{square()} liczącej kwadrat zmiennej każdego typu\\
\vspace{0.1cm}
\begin{lstlisting}
int square(int x){  return x*x;}
short int square(short int x){  return x*x;}
unsigned short int square(unsigned short int x){  return x*x;}
long square(long x){  return x*x;}
unsigned long square(unsigned long x){  return x*x;}
long long square(long long x){  return x*x;}
unsigned long long square(unsigned long long x){  return x*x;}
float square(float x){  return x*x;}
double square(double x){  return x*x;}
// i tak dalej, i jeszcze dla naszych typow ...
\end{lstlisting}

\end{frame}

\begin{frame}[fragile]
  \frametitle{Bez szablonów funkcji}
  \framesubtitle{Po co? przykład z przeładowaniem}

A potem dodamy jeszcze dla wszystkich klas jakie tworzymy...
\vspace{0.1cm}
\begin{lstlisting}
class complex{//liczba zespolona na int
  public:
  complex(int r, int i) : real(r), imag(i) {}
  const int& c_real(){return real;}
  const int& c_imag(){return imag;}
  private:
  int real;
  int imag;
};

complex square_elements(complex& a){
  return complex(a.c_real()*a.c_real(),
                a.c_imag()*a.c_imag());
}
\end{lstlisting}
A może by tak napisać program do pisania programu?!
\end{frame}

\begin{frame}[fragile]
  \frametitle{Szablon funkcji}
  \framesubtitle{Składnia}
Zamiast pisać dużo kodu możemy napisać 'ogólną receptę':\\
Tak to język w języku ...
\vspace{0.1cm}
\begin{lstlisting}
template <class T> // l typename
typ_funkcji fun(argumenty)
{
  Cialo funkcji
}
\end{lstlisting}  
W naszym przypadku:\\
\begin{lstlisting}
template <class T> 
T square(T x)
{
  return x*x; // T musi wiedziec co to '*'
}
\end{lstlisting}  
\end{frame}

\begin{frame}[fragile]
  \frametitle{Szablon funkcji}
  \framesubtitle{Składnia}
\begin{lstlisting}
template <class T> 
T square(T x)
{
  return x*x; // T musi wiedziec co to '*'
}

int a=9;
square<int>(a);
\end{lstlisting}  
nasza klasa \textit{complex} nie wie co to * ...

\end{frame}

\begin{frame}[fragile]
  \frametitle{Szablon funkcji}
  \framesubtitle{Domyślny kompilator}
\begin{lstlisting}
template <class T> 
T square(T x)
{
  return x*x; // T musi wiedziec co to '*'
}

double b=9;//kompilator 'domysli sie typu
square(b);
\end{lstlisting}  

W przypadku szablonów funkcji można pominąć $<>$ kompilator postara się zdecydować za nas. No chyba, że nie może ... na przykład:
\vspace{0.1cm}
\begin{lstlisting}
template<typename T>
T my_type_caster_no_sense(double x)
{
  T a = x; //rzutowanie na T
  return a;
}
\end{lstlisting}

\end{frame}

\begin{frame}[fragile]
  \frametitle{Szablon funkcji}
  \framesubtitle{Argumenty szablonu}
  \begin{itemize}
    \item Typy, \textit{class} lub \textit{typename}
    \item Wartości całkowite, \textit{int}, \textit{enum} ...
  \end{itemize}
  
\begin{lstlisting}
template <class T> //typename
T square(T x)
{
  return x*x; // T musi wiedziec co to '*'
}

template<int V>
int templated_with_value(double x)
{
  return a*V;
}
\end{lstlisting}
  
\end{frame}

\begin{frame}[fragile]
  \frametitle{Szablon funkcji}
  \framesubtitle{Argumenty szablonu - może być kilka}
  
  \begin{lstlisting}
template <class T, int N> //typename
T pow(T x)
{
  T res=1;
  for(int i=0; i<N; ++i)
    res *= x;
  return res;
}
\end{lstlisting}

\begin{lstlisting}
template <int N, class T> //typename
T pow(T x)
{
  T res=1;
  for(int i=0; i<N; ++i)
    res *= x;
  return res;
}
\end{lstlisting}
  
\end{frame}

\begin{frame}[fragile]
  \frametitle{Specjalizacja}
  \framesubtitle{Czyli co robić dla określonych argumentów szablonu}

\begin{lstlisting}
template<typename T>
T fun(T x)
{
  return 2*x;
}

template<>
string fun<string>(string x)
{
  return x + " " +x;
}


int main()
{
  double a=9;
  cout << fun(a) << endl;
  
  string b="Ala ma kota";
  cout << fun(b) << endl;
}
\end{lstlisting}
\end{frame}

\subsection{Klasy}

\begin{frame}[fragile]
  \frametitle{Szablon klasy}
  \framesubtitle{Przykład}
\begin{lstlisting}
class complex_int{//liczba zespolona na int
  public:
  complex_int(int r, int i) : real(r), imag(i) {}
  const int& c_real(){return real;}
  const int& c_imag(){return imag;}
  private:
  int real;
  int imag;
};
\end{lstlisting}
\begin{lstlisting}
class complex_double{//liczba zespolona na int
  public:
  complex_double(double r, double i) : real(r), imag(i) {}
  const double& c_real(){return real;}
  const double& c_imag(){return imag;}
  private:
  double real;
  double imag;
};
\end{lstlisting}
A inne typy? A kontenery?
\end{frame}

\begin{frame}[fragile]
  \frametitle{Szablon klasy}
  \framesubtitle{Składnia}
  
\begin{lstlisting}
template <class T>
class class_name{//cialo oparte o T
  pyblic:
  class_name(){}
  void fun();
  private:
  T a;
  T b;
};

template<class T>
void class_name<T>::fun()
{
...
}
\end{lstlisting}

\end{frame}

\begin{frame}[fragile]
  \frametitle{Szablon klasy}
  \framesubtitle{Różne!}
  
\begin{lstlisting}
template <class T>
class class_name{//cialo oparte o T
  pyblic:
  class_name(){}
  void fun();
  private:
  T a;
  T b;
};

...
class_name<int> != class_name<double>!!!
\end{lstlisting}

\end{frame}

\begin{frame}[fragile]
  \frametitle{Szablon klasy}
  \framesubtitle{Argumenty}
    \begin{itemize}
    \item Typy, \textit{class} lub \textit{typename}
    \item Wartości całkowite, \textit{int}, \textit{enum} ...
  \end{itemize}
  
\begin{lstlisting}
template <class T>
class class_name{//cialo oparte o T
  pyblic:
  class_name(){}
  void fun();
  private:
  T a;
  T b;
};

template<int V>
class class_name2{
  pyblic:
  class_name(){}
  void fun();
  private:
  int a[V];
  double b[V];
  
};
\end{lstlisting}
  
\end{frame}

\begin{frame}[fragile]
  \frametitle{Szablon klasy}
  \framesubtitle{Argumenty szablonu - może być kilka}
  
  \begin{lstlisting}
template <class T, int N> //typename
class class_name{//cialo oparte o T
  pyblic:
  class_name(){}
  void fun();
  private:
  T tab[N];
};
\end{lstlisting}

\begin{lstlisting}
template <int N, class T> //typename
class class_name{//cialo oparte o T
  pyblic:
  class_name(){}
  void fun();
  private:
  T tab[N];
};
\end{lstlisting}
  
\end{frame}

\begin{frame}[fragile]
  \frametitle{Szablon klasy}
  \framesubtitle{Specjalizacja, czyli co robić dla określonych argumentów szablonu}
\begin{lstlisting}
template<int N>
class foo{
  public:
  foo();
  void fun();
  double tab[N];
};

template<int N>
foo<N>::foo()
{}

template<>
foo<0>::foo()
{
  cout << "This will be empty!!!" << endl;
}
\end{lstlisting}
\end{frame}

\section{Metaprogramowanie}

\begin{frame}[fragile]
  \frametitle{ZŁO!}
  \framesubtitle{Język w języku}
  \begin{itemize}
    \item Okazuje się, że mechanizm szablonów w C++ jest osobnym językiem
    \item Tak, wewnątrz C++ zaszyto inny język
    \item Turing Kompletny ...
    \item Odkryto to przypadkiem
    \item I tak, ma zastosowania
    \item Pozwala na wykonanie "programu" w czasie kompilacji - metaprogramowanie!
  \end{itemize}
\end{frame}

\begin{frame}[fragile]
  \frametitle{Przykład}
  \framesubtitle{Silnia - klasycznie}
\begin{lstlisting}
unsigned int factorial(unsigned int n)
{
  cout << "Called with n=" << n << endl;
  return n == 0 ? 1 : n * factorial(n - 1); 
}

int main()
{
  cout <<" Finally the factorial is:" << factorial(4) << endl;  
}
\end{lstlisting}
\end{frame}

\begin{frame}[fragile]
  \frametitle{Przykład}
  \framesubtitle{Silnia - inaczej}
\begin{lstlisting}
template <unsigned int n>
class factorial {
public:
  static const unsigned long long value = n * factorial<n-1>::value;
};

template <>
class factorial<0> {
public:
  static const unsigned long long value = 1;
};


int main()
{
  std::cout << factorial<5>::value << "\n";
}
\end{lstlisting}
\end{frame}

\begin{frame}[fragile]
  \frametitle{Częściowa Specjalizacja}

\begin{lstlisting}
template<int N, class T>
class foo{
  public:
  foo();
  void fun();
  T tab[N];
};
template<class T> //specialized for N=0
class foo<0,T>{
  public:
  foo();
  void fun();
  T tab[0];
};
template<int N, class T>
foo<N,T>::foo()
{}
template<class T>
foo<0,T>::foo()
{
  cout << "This will be empty!!!" << endl;
}
\end{lstlisting}
\end{frame}
\end{document}
