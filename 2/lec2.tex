%\documentclass[draft,10pt]{beamer}
\documentclass[10pt]{beamer}

\usepackage[english,polish]{babel}
\usepackage[utf8]{inputenc}
\usepackage{polski}

\usepackage{define}

\eventtitle{Programowanie obiektowe w języku C++}
\title{Programowanie obiektowe w języku C++}
\author[shortname]{Stanis{\l}aw Gepner}
\institute[shortinst]{sgepner@meil.pw.edu.pl}
\date{}

\setbeamertemplate{blocks}[rounded][shadow=true]
\setbeamertemplate{navigation symbols}[]

\begin{document}

\frame{
    \titlepage
}

\section{Wspomnienia}

\begin{frame}
  \frametitle{Co już wiemy}
  
  \centering
  \begin{enumerate}
    \item Proceduralnie a obiektowo
    \item Obiekt to instancja klasy, łączy dane i metody
    \item Co ++ daje C? - nowe elementy języka
    \begin{itemize}
      \item namespace
      \item referencja \&
      \item new delete
      \item ...
    \end{itemize}
    \item Coś wspominaliśmy o klasach ...
  \end{enumerate}
\end{frame}

\section{Klasy (i struktury)}

\begin{frame}[fragile]
  \frametitle{Struktura}
  \framesubtitle{C}
  \centering
  \begin{itemize}
    \item Dostępne w C, \textit{struct\{\};}.
    \item Pozwalają na grupowanie danych, brak metod (prawie)
    \item Definiuje nowy typ zmiennej, zmienne tego typu można deklarować
    \item Brak hermetyzacji
  \end{itemize}
  
  \begin{lstlisting}
#include <stdio.h>

struct student{
  int numerindeksu;
  float ocenazcpp;
  void  (*pprint)(struct student*);
};
void print(struct student* self)
{
  printf("Student %d recived %1.1f \n ", self->numerindeksu, self->ocenazcpp);
}
int main(){
  struct student s1;
  s1.pprint = print;
  s1.numerindeksu = 207778;
  s1.ocenazcpp = 2;
  s1.pprint(&s1);
}
\end{lstlisting}
\end{frame}

\begin{frame}[fragile]
  \frametitle{Struktura}
  \framesubtitle{C++}
  \centering
  \begin{itemize}
    \item Rozszerzone możliwości dostępne w C++
    \item Może mieć metody
    \item Dostępna kontrola dostępu - hermetyzacja
    \item Domyślnie wszystko \textit{public}, w klasie zaś, \textit{private}
  \end{itemize}
  
\begin{lstlisting}
#include <iostream>
using namespace std;
struct student{
  int numerindeksu;
  void setOcena(float o){ocenazcpp=o;}
  void print(void)
  {
    cout << "Student " << numerindeksu << " recived " << ocenazcpp << endl;
  }
  float ocenazcpp;
};

int main(){
  student s1;
  s1.numerindeksu = 207778;
  s1.setOcena(2.5);
  
  s1.print();
}
\end{lstlisting}
\end{frame}

\begin{frame}[fragile]
  \frametitle{Klasa}
  \framesubtitle{\textit{class\{...\}}}
  \centering
  
  \begin{itemize}
    \item Klasa to definicja dla \textit{obiektu}, a obiekt to \textit{instancja} klasy.
    \item Klasa opisuje dane oraz interfejs.
    \item Interfejs to opis sposobu komunikacji z instancjami klasy.
    \item Podobna do struktury
  \end{itemize}

\begin{lstlisting}
#include <iostream>
using namespace std;
class student{
  public:
    int numerindeksu;
    void setOcena(float o){ocenazcpp=o;}
  void print(void)
  {
    cout << "Student " << numerindeksu << " recived " << ocenazcpp << endl;
  }
  private:
    float ocenazcpp;
};
int main(){
  student s1;
  s1.numerindeksu = 207778;
  s1.setOcena(2.5);
  
  s1.print();
}
\end{lstlisting}
\end{frame}

\begin{frame}[fragile]
  \frametitle{Klasa}
  \begin{columns}
    \begin{column}{0.48\textwidth}
\begin{lstlisting}
class foobar{
//class body

//atrybuty

//metody
...
};

//metody cd.

\end{lstlisting}
    \end{column}
    \begin{column}{0.48\textwidth}
      \begin{itemize}
        \item Definicja nowej klasy zaczyna się od słówka \textit{class}
        \item Dalej nazwa - identyfikator
        \item Ciało klasy zawarte jest między \{ \} zakończone ;
        \item Atrybuty i metody w ciele klasy
        \item Ciało metod może być poza klasą
      \end{itemize}
    \end{column}
  \end{columns}
\end{frame}

\begin{frame}[fragile]
  \frametitle{Klasa}
  \begin{columns}
    \begin{column}{0.48\textwidth}
\begin{lstlisting}
class foobar{
public:
  int foo;
  int fun(){return foo; }
};

foobar fu;
fu.foo = 5;
int a = fu.fun();

foobar *p = new foobar;
p->foo;
int b = p->fun();

delete p;

\end{lstlisting}
    \end{column}
    \begin{column}{0.48\textwidth}
      \begin{itemize}
        \item Z zewnątrz dostęp przez .\\
        np.: \textit{foobar.foo}\\
        lub \textit{foobar.fun()}
        \item lub przez -$>$ jeżeli wskaźnik\\
        np.: \textit{p-{$>$}foo}\\
        lub \textit{p-{$>$}fun()}
        \item Metody klasy mają dostęp do wszystkiego\\
        przez nazwę \textit{return foo;}\\
        lub przez \textit{this-{$>$}foo1;}
        \item Atrybutami mogą być typy proste, inne klasy, kolekcje, wskaźniki i referencje ...
      \end{itemize}
    \end{column}
  \end{columns}
\end{frame}

\begin{frame}[fragile]
  \frametitle{Funkcje w klasie? Czyli metody!}
  \begin{columns}
    \begin{column}{0.48\textwidth}
\begin{lstlisting}
#include <iostream>

using namespace std;

class person{
  public:
    void setAge(int a){ mAge=a; }
    int getAge(){ return mAge; }
    void printtS(){cout << mS << endl;}
    void calcS();
  private:
    int mAge;
    int mS;
};

void person::calcS(){
  mS = 2 * mAge;
}

int main(){
  person p;
  p.setAge(3);
  p.calcS();
  cout << p.getAge() << endl;
  p.printtS();
}


\end{lstlisting}
    \end{column}
    \begin{column}{0.48\textwidth}
      \begin{itemize}
        \item Mogą być w ciele
        \item albo poza - z deklaracją w ciele i operatorem zasięgu ::
        \item w .h lub .cpp
        \item Metoda ma dostęp do wszystkich atrybutów klasy  
      \end{itemize}
    \end{column}
  \end{columns}
\end{frame}

\begin{frame}[fragile]
  \frametitle{Metody}
  \frametitle{Domyślne argumenty}
  \begin{columns}
    \begin{column}{0.48\textwidth}
\begin{lstlisting}
#include <iostream>
using namespace std;
class person{
  public:
    void setAge(int a = 0)
    {mAge=a;}
    void setS(int s=0);
    void print()
    {
      cout << mAge << " " << mS << endl;
    }
  private:
    int mAge;
    int mS;
};

void person::setS(int s){
  mS = s;
}

int main(){
  person p;
  p.setAge();
  p.setS();
  p.print();
}


\end{lstlisting}
    \end{column}
    \begin{column}{0.48\textwidth}
      \begin{itemize}
        \item Na końcu listy argumentów
        \item Jeżeli ciało w klasie to OK
        \item Jeżeli metoda ma ciało poza klasą, to tylko w prototypie w klasie
      \end{itemize}
    \end{column}
  \end{columns}
\end{frame}

\begin{frame}[fragile]
  \frametitle{Klasa}
  \framesubtitle{public: i private:}
  \begin{columns}
    \begin{column}{0.48\textwidth}
\begin{lstlisting}
#include <iostream>
using namespace std;
class foobar{
public:
  int foo;
  int fun(){return foo1; }
  int fun_other(foobar& rf)
  {
    return rf.fun1();
  }
private:
  int foo1;
  int fun1(){
    return this->foo;
  }
};

int main()
{
  foobar fu1, fu2;
  fu1.foo = 5;
  fu2.foo = 9;
  
  int a = fu1.fun_other(fu1);
  int b = fu1.fun_other(fu2);
}
\end{lstlisting}
    \end{column}
    \begin{column}{0.48\textwidth}
      \begin{itemize}
        \item \textit{public:} i \textit{private:} określają widoczność, dostęp do członków klasy; atrybutów i metod
        \item Definiują sekcje, wszystko w sekcji ma daną widoczność.
        \item \textit{public:} nieograniczony dostęp z każdego miejsca i przez wszystko
        \item \textit{private:} tylko metody instancji danej klasy mają dostęp
        \item ... To znaczy inne obiekty tej samej klasy też ...
        \item Kontrola w czasie kompilacji nie wykonania
      \end{itemize}
    \end{column}
  \end{columns}
\end{frame}

\begin{frame}[fragile]
  \frametitle{Klasa}
  \framesubtitle{public: i private:}
  \begin{columns}
    \begin{column}{0.48\textwidth}
\begin{lstlisting}
#include <iostream>
using namespace std;
class foobar{
public:
  \\ metody
private:
  \\ dane 
  \\ metody
};

\end{lstlisting}
    \end{column}
    \begin{column}{0.48\textwidth}
      \begin{itemize}
        \item dane prywatne
        \item metody publiczne
        \item Bezpieczeństwo przed niepożądaną manipulacją
        \item Ukrywaj atrybuty, wystawiaj interfejs
        \item Jak bankomat
      \end{itemize}
    \end{column}
  \end{columns}
\end{frame}

\begin{frame}[fragile]
  \frametitle{Klasa}
  \framesubtitle{public: i private:}
  \begin{columns}
    \begin{column}{0.48\textwidth}
\begin{lstlisting}
#include <iostream>
using namespace std;
class foobar{
public:
  void setFoo(int f){foo=f;}
  int getFoo(){return foo;}
  const int& cFoo(){return foo;}
  int& rFoo(){return foo;}
private:
  int foo;
};

int main()
{
  foobar fu1,;
  ...
}
\end{lstlisting}
    \end{column}
    \begin{column}{0.48\textwidth}
      \begin{itemize}
        \item atrybut int foo jest prywatny
        \item dodano metody get/set do operowania
        \item nowe słówko const
        \item metoda \textit{const int\& cFoo()} zwraca referencje
        \item metofa \textit{int\& rFoo()} też ...
        \item Różnica? patrz przykład
        \item Jak nie zapomnimy to wrócimy
      \end{itemize}
    \end{column}
  \end{columns}
\end{frame}

\begin{frame}[fragile]
  \frametitle{Tworzenie obiektu}
  \framesubtitle{Konstruktor}
  \begin{columns}
    \begin{column}{0.48\textwidth}
\begin{lstlisting}
#include <iostream>
class foobar{
  private:
    int a;
    int b;
};
int main(){
  std::cout << sizeof(foobar) << std::endl;
  foobar f1; //obiekt f1 typu foobar isnieje
  std::cout << sizeof(foobar) << std::endl;
}
\end{lstlisting}
    \end{column}
    \begin{column}{0.48\textwidth}
      \begin{itemize}
        \item Domyślny, parametryczny, kopiujący, lista inicjalizacyjna ...
        \item Konstruktor to specjalna metoda klasy
        \item Jest wywoływana przy tworzeniu obiektu
        \item Pamięć alokowana w chwili powołania instancji klasy
        \item Jeżeli go nie stworzymy zrobi to kompilator
      \end{itemize}
    \end{column}
  \end{columns}
\end{frame}

\begin{frame}[fragile]
  \frametitle{Tworzenie obiektu}
  \framesubtitle{Konstruktor domyślny}
  \begin{columns}
    \begin{column}{0.48\textwidth}
\begin{lstlisting}
#include <iostream>
class foobar{
  public:
    foobar() {}
  private:
    int a;
    int b;
};
int main(){
  foobar f1;
}
\end{lstlisting}
    \end{column}
    \begin{column}{0.48\textwidth}
      \begin{itemize}
        \item Jest zawsze
        \item Powinien być w sekcji public:
        \item ... chyba, że nie chcemy ...
        \item składnia:\\ \textit{identyfikator() \{ ciało \}}
        \item Jak każda metoda może być w lub poza ciałem klasy
        \item Może mieć niepuste ciało
        \item Przykład ...
      \end{itemize}
    \end{column}
  \end{columns}
\end{frame}

\begin{frame}[fragile]
  \frametitle{Tworzenie obiektu}
  \framesubtitle{Konstruktor parametryczny}
  \begin{columns}
    \begin{column}{0.48\textwidth}
\begin{lstlisting}
#include <iostream>
class foobar{
  public:
    foobar() {}
    foobar(int aa, int bb=9)
    {
      a = aa;
      b = bb;
    }
  private:
    int a;
    int b;
};
int main(){
  foobar f1, f2(3), f3(3,4);
}
\end{lstlisting}
    \end{column}
    \begin{column}{0.48\textwidth}
      \begin{itemize}
        \item Służy do przekazania argumentów
        \item składnia:\\ \textit{identyfikator(argumenty)\\ \{ ciało \}}
        \item Jak każda metoda może być w lub poza ciałem klasy
        \item Może mieć domyślne parametry
        \item Przykład ...
      \end{itemize}
    \end{column}
  \end{columns}
\end{frame}

\begin{frame}[fragile]
  \frametitle{Tworzenie obiektu}
  \framesubtitle{Lista inicjalizacyjna}
  \begin{columns}
    \begin{column}{0.48\textwidth}
\begin{lstlisting}
#include <iostream>
class foobar{
  public:
    foobar(){}
    foobar(int aa, int bb=9)
    : a(aa), b(bb) { }
  private:
    int a;
    int b;
};
int main(){
  foobar f1, f2(3), f3(3,4);
  int c(9); //konsekwencja
}
\end{lstlisting}
    \end{column}
    \begin{column}{0.48\textwidth}
      \begin{itemize}
        \item Ustawia wartość w czasie tworzenia obiektu
        \item składnia:\\ \textit{identyfikator(argumenty) : var(val)\\ \{ ciało \}}
        \item przydatne gdy obiekt przechowuje referencje
        \item Czasem nie można w \{\}
      \end{itemize}
    \end{column}
  \end{columns}
\end{frame}

\begin{frame}[fragile]
  \frametitle{Tworzenie obiektu}
  \framesubtitle{Konstruktor kopiujący}
  \begin{columns}
    \begin{column}{0.48\textwidth}
\begin{lstlisting}
#include <iostream>
class foobar{
  public:
    foobar(){}
    foobar(const foobar& f)
    {
      a = f.a;
      b = f.b
    }
  private:
    int a;
    int b;
};
int main(){
  foobar f1, f2(3), f3(3,4);
  int c(9); //konsekwencja
}
\end{lstlisting}
    \end{column}
    \begin{column}{0.48\textwidth}
      \begin{itemize}
        \item Jest zawsze (?)
        \item Powinien być w sekcji public:
        \item ... chyba, że nie chcemy ...
        \item składnia:\\ \textit{identyfikator(iden\& i)
        \\ \{ ciało \}}
        \item Jak każda metoda może być w lub poza ciałem klasy
        \item Może mieć niepuste ciało
		    \begin{itemize}
		    	\item Przy przekazywaniu z i do funkcji
	      \end{itemize}		    
        \item Przykład ...
      \end{itemize}
    \end{column}
  \end{columns}
\end{frame}

\begin{frame}[fragile]
  \frametitle{Niszczenie obiektu}
  \framesubtitle{Destruktor}
  \begin{columns}
    \begin{column}{0.48\textwidth}
\vspace{-0.2cm}
\begin{lstlisting}
#include <iostream>
#include <stdlib.h>

using namespace std;

class collection{
public:
  collection(){size=0; tab=NULL;}
  collection(int s) : size(s) {allocate();}

  ~collection(){
    cout << "The cleaning service!" << endl;
    delete []tab;
  }
  void setSize(int a){ size=a; }
  int getSize(){ return size; }
  void allocate();
  int& rTab(int i)
  { return tab[i];}
private:
  int size;
  int * tab;
};
\end{lstlisting}
    \end{column}
    \begin{column}{0.48\textwidth}
\begin{lstlisting}
void collection::allocate()
{
  tab = new int[size];
}
\end{lstlisting}
      \begin{itemize}
        \item Odwrotność konstruktora
        \item Wywoływany przy końcu życia obiektu
        \item Nigdy nie wywoływany jawnie
        \item Jest tylko jeden
        \item Metody specjalna, tworzona domyślnie przez kompilator
        \item Definicja w lub poza ciałem
        \item \textit{{$\sim$} identyfikator(argumenty)\\ \{ ciało \}}
        \item Destruktor wywoływany przed zwolnieniem zasobów
      \end{itemize}
    \end{column}
  \end{columns}
\end{frame}

% \begin{frame}[fragile]
%   \frametitle{Zmienne statyczne}
%   \begin{columns}
%     \begin{column}{0.48\textwidth}
% \vspace{-0.2cm}
% \begin{lstlisting}
% class foobar{
%   public:
%     static int a;
%     int b = 5; // -std=c++11
% };
% int main(){
%   foobar f1, f2;
%   ...
% }
% \end{lstlisting}
%     \end{column}
%     \begin{column}{0.48\textwidth}
%       \begin{itemize}
%         \item Nowe słówko \textit{static}
%         \item Współdzielone przez wszystkie instancje klasy
%         \item Inicjalizacja poza ciałem bez słówka \textit{static}
%         \item Można użyć bez instancji klasy!
%         \item Można zmienić wartość
%         \item Przykład:
%       \end{itemize}
%     \end{column}
%   \end{columns}
% \end{frame}
%
% \begin{frame}[fragile]
%   \frametitle{Zmienne statyczne const ...}
%   \begin{columns}
%     \begin{column}{0.48\textwidth}
% \vspace{-0.2cm}
% \begin{lstlisting}
% class foobar{
%   public:
%     static const int a;
%     int b = 5; // -std=c++11
% };
% int main(){
%   foobar f1, f2;
%   ...
% }
% \end{lstlisting}
%     \end{column}
%     \begin{column}{0.48\textwidth}
%       \begin{itemize}
%         \item \textit{static const}
%         \item Współdzielone przez wszystkie instancje klasy
%         \item Inicjalizacja w ciele
%         \item Można użyć bez instancji klasy!
%         \item Nie można zmienić wartości
%         \item Przykład:
%       \end{itemize}
%     \end{column}
%   \end{columns}
% \end{frame}
%
% \begin{frame}[fragile]
%   \frametitle{Funkcje statyczne}
%   \begin{columns}
%     \begin{column}{0.48\textwidth}
% \vspace{-0.2cm}
% \begin{lstlisting}
% #include <iostream>
% using namespace std;
% class foobar{
%   public:
%     static int a;
%     int b;
%     
%     static void static_do(int new_d)
%     {
%       cout << "a=" << a << endl;
%       a = new_d;
%       //b = a; // nie wolno
%     }
% };
% int foobar::a = 9;
% int main(){
%   foobar f1, f2, f3;
%   cout << f1.a << endl;
%   foobar::static_do(9);
%   cout << f3.a << endl;
% }
% \end{lstlisting}
%     \end{column}
%     \begin{column}{0.48\textwidth}
%       \begin{itemize}
%         \item \textit{static typ nazwa()\{\}}
%         \item Współdzielone przez wszystkie instancje klasy
%         \item Można użyć bez instancji klasy!
%         \item Może działać tylko na zmiennych, które są \textit{static}
%         \item Przykład:
%       \end{itemize}
%     \end{column}
%   \end{columns}
% \end{frame}
%
% \section{Operatory}
%
% \begin{frame}[fragile]
%   \frametitle{Operatory}
%       \begin{itemize}
%         \item Operator to może być symbol, np.: +, -, *, /
%         \item albo \textit{new new[] delete []delete}
%         \item albo konwersja typów (przemilczeć)
%         \item W C++ użycie operatora jest on związane w wywołaniem metody
%         \item C++ pozwala na definiowanie własnych operatorów
%         \item Nie dla typów wbudowanych
%         \item Definicje mogą być w ciele, albo poza
%         \item Jedno i dwu argumentowe
%         \item W zasadzie metoda może być dowolna
%         \item Operator powinien wykonywać przewidywalną operację np: \\
%         c=a+b nie powinien zmieniać a, b
%         \item Jest tego dużo
%       \end{itemize}
% \end{frame}
%
% \begin{frame}[fragile]
%   \frametitle{Operatory}
%   \framesubtitle{Lista niepełna}
%           + operator dodawania, może być jedno lub dwuargumentowy\\
%           - operator odejmowania, może być jedno lub dwuargumentowy\\
%           * operator mnożenia, dwuargumentowy\\
%           / operator dzielenia, dwuargumentowy\\
%           \% operator dwuargumentowy\\
%           \& jedno lub dwu argumentowy\\
%           = operator przypisania, może być domyślny\\
% $\sim$, !, =, 
% $<$, $>$
% +=, -=
% *=, /=
% \%=, \&=
% |=, <<
% >>, >>=
% <<=, ==
% !=, $<$=
% $>$=, \&\&
% ||, ++
% --, ,, -$>$*, -$>$, [], \\
% () - wywołania - ile chcemy argumentowy\\
% poniższe operatory gdy ich nie zdefiniujemy robi to za nas kompilator
% new, new[], delete, delete[]
%
% \end{frame}
%
% \begin{frame}[fragile]
%   \frametitle{Operator}
%   \framesubtitle{Na razie bez sensu}
%   \begin{columns}
%     \begin{column}{0.48\textwidth}
% \begin{lstlisting}
% //Ogolna skladnia
% typ operator@ (argumenty)
% {
% // operacje 
% }
%
% //wywolanie
% operator@(args);
% \end{lstlisting}
%     \end{column}
%     \begin{column}{0.48\textwidth}
%     \begin{lstlisting}
% #include <iostream>
% using namespace std;
% class foo{
%   public:
%   int a;
% };
% void operator+(foo a) //zle
% {
%   a.a*=-1;
%   cout << "Wywolano operator + a=" << a.a << endl;
% }
% void operator*(foo& a)//tez zle
% {
%   a.a*=-1;
%   cout << "Wywolano operator - a=" << a.a << endl;
% }
% int main(){
%   foo f1;
%   f1.a = 9;
%   cout << f1.a << endl;
%   operator*(f1); //to tez jest wywolanie
%   +f1;
%   *f1;
%   cout << f1.a << endl;
% }
% \end{lstlisting}
%     \end{column}
%   \end{columns}
% \end{frame}
%
% \begin{frame}[fragile]
%   \frametitle{Operator dwuargumentowy}
%
% \begin{lstlisting}
% #include <iostream>
% using namespace std;
% class foo{
%   public:
%   int a;
% };
% void operator+(foo& a, foo& b) //nie zwraca??
% {
%   a.a = a.a + b.a;
%   cout << "Wywolano operator + a=" << a.a << endl;
% }
%
% int main(){
%   foo f1, f2;
%   f1.a = 9;
%   f2.a = 3;
%   cout << f1.a << endl;
%   f1+f2;
%   cout << f1.a << endl;
% }
% \end{lstlisting}
% Na razie trochę bez sensu, więcej przykładów ...
% \end{frame}
%
% \begin{frame}[fragile]
%   \frametitle{Operator =}
%   \framesubtitle{Teraz lepiej}
%   
%   \begin{columns}
%     \begin{column}{0.48\textwidth}
% \begin{lstlisting}
% class Foo {
% public:
% ...
%   Foo & operator=(const Foo &prawa);
% ...
% }
%
% Foo& Foo::operator=(const Foo &p)
% {
%   if (this != &p)// Samo kopia?
%   {
%    ... // Wykonaj
%   }
%   return *this; //zwroc siebie
% }
%
% Foo a, b;
% ...
% b = a;   // Jak b.operator=(a);
% \end{lstlisting}
%     \end{column}
%     \begin{column}{0.48\textwidth}
%       \begin{itemize}
%         \item W ciele klasy
%         \item Argument (prawa strona) jest \textit{const}
%         \item Operator zwraca by umożliwić a=b=c
%         \item Samo przypisanie? a=a\\
%         \textit{if (this != \&rhs)}
%         \item przykład ...
%       \end{itemize}
%     \end{column}
%   \end{columns}
%   
% \end{frame}
%
% \begin{frame}[fragile]
%   \frametitle{Operator +=, -=, *=}
%   \framesubtitle{Teraz lepiej}
%   
%   \begin{columns}
%     \begin{column}{0.48\textwidth}
% \begin{lstlisting}
% class Foo {
% public:
% ...
%   Foo & operator+=(const Foo &prawa);
% ...
% }
%
% Foo& Foo::operator+=(const Foo &p)
% {
%   if (this != &p)// a+=a?
%   {
%    ... // Wykonaj
%   }
%   return *this; //zwroc siebie
% }
%
% Foo a, b;
% ...
% b += a;   // Jak b.operator+=(a);
% \end{lstlisting}
%     \end{column}
%     \begin{column}{0.48\textwidth}
%       \begin{itemize}
%         \item W ciele lub poza klasy
%         \item Argument (prawa strona) jest \textit{const}
%         \item Operator zwraca by umożliwić a+=b+=c ?
%         \item przykład ...
%       \end{itemize}
%     \end{column}
%   \end{columns}
%   
% \end{frame}
%
% \begin{frame}[fragile]
%   \frametitle{Operator dwuargumentowy}
%   \framesubtitle{W ciele}
%
%     \begin{lstlisting}
% #include <iostream>
% using namespace std;
% class Foo{
%   public:
%   const Foo operator+(const Foo &prawy) const {
%     Foo res = *this;  // Kopia res(*this);
%     res += prawy;     // Uzyj +=
%     return res;       // !
%   }
% };
%
% int main(){
%   Foo a, b, c;
%   
%   c = a+b;
% \end{lstlisting}
% \end{frame}
%
% \begin{frame}[fragile]
%   \frametitle{Operator dwuargumentowy}
%   \framesubtitle{Poza ciałem - friend}
%
%     \begin{lstlisting}
% #include <iostream>
% using namespace std;
% class foo{
%   public:
%   foo(int a) : a(a) {}
%   void print() { cout << a << endl; }
%   private:
%   int a;
%   friend foo operator+(const foo& a, const foo& b);
% };
% foo operator+(foo& a, foo& b)
% {
%   foo tmp = a;
%   tmp.a += b.a;
%   return tmp;
% }
%
% int main(){
%   foo f1(3), f2(2);
%   f1.print();
%   f2.print();
%   foo f3 = f1+f2;
%   f3.print();
% }
% \end{lstlisting}
% \end{frame}

\end{document}
